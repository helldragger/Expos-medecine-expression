\documentclass{beamer}

\usetheme{keynote-vintage}

\usepackage{tikz}
\usetikzlibrary{shapes,arrows}

%\usepackage{fontspec}

\usepackage[utf8]{inputenc}
\usepackage[T1]{fontenc}
\usepackage{dejavu}

\usepackage{helvet}
\renewcommand{\familydefault}{\sfdefault}

\definecolor{textcolour}{rgb}{0.37,0.34,0.27}
\definecolor{davinci}{RGB}{148,77,23}

\setbeamertemplate{background canvas}{\includegraphics [width=\paperwidth, height=\paperheight]{pics/Vintage.png}}

\setbeamercolor{structure}{fg=textcolour}
\setbeamertemplate{items}[circle]
\setbeamerfont{structure}{family = \rmfamily}



\setbeamerfont{title}{size = \large}
\setbeamerfont{frametitle}{size = \tiny}


\usefonttheme{structuresmallcapsserif}
\usefonttheme{serif}
\setbeamercolor{normal text}{fg=textcolour}

\renewcommand{\tiny}{\fontsize{6pt}{7pt}\selectfont}
\renewcommand{\scriptsize}{\fontsize{7pt}{10pt}\selectfont}
\renewcommand{\footnotesize}{\fontsize{8pt}{10pt}\selectfont}
\renewcommand{\small}{\fontsize{10pt}{14pt}\selectfont}
\renewcommand{\normalsize}{\fontsize{10pt}{12pt}\selectfont}
\renewcommand{\large}{\fontsize{12pt}{20pt}\selectfont}
\renewcommand{\Large}{\fontsize{20pt}{33pt}\selectfont}
\renewcommand{\LARGE}{\fontsize{32pt}{43pt}\selectfont}
\renewcommand{\huge}{\fontsize{43pt}{50pt}\selectfont}
\renewcommand{\Huge}{\fontsize{76pt}{92pt}\selectfont}

\setbeamerfont{enumerate item}{size=\LARGE}
\setbeamerfont*{quote}{size=\large,shape=\itshape,series=\bfseries}
\setbeamerfont{word frame}{size=\Large, family=\rmfamily}

\newcommand{\titleframe}{
	\setbeamertemplate{background canvas}{\includegraphics [width=\paperwidth, height=\paperheight]{pics/fond.jpg}}
	
	\begin{frame}
		\maketitle
	\end{frame}
	\setbeamertemplate{background canvas}{\includegraphics [width=\paperwidth, height=\paperheight]{pics/Vintage.png}}
}


%%% An imageframe has one fullscreen image as background
%%% and maybe some text on top.
\newcommand{\imageframe}[2]{
%	\setbeamertemplate{background}{%
%		\parbox[c][\paperheight]{\paperwidth}{%
%			\includegraphics[width=\paperwidth,height=\paperheight]{#1}
%	}}
	\begin{frame}
		\begin{figure}[ht]
			\includegraphics[width=0.9\linewidth, height=0.8\paperheight]{#1}
			\caption{\large\usebeamercolor[fg]{normal text}\centering#2}
		\end{figure}
	\end{frame}
}

%%% A wordframe has one word (or few) big and centered
\newcommand{\wordframe}[1]{
	\begin{frame}
		\bf\centering\usebeamerfont{word frame}#1
	\end{frame}
}

%%% A defnframe defines a word or phrase
\newenvironment{defnframe}[1]
{
	\begin{frame}
		\usebeamerfont{title}\usebeamercolor[fg]{box title}\Large #1: \\
		\vskip0.7cm
		\usebeamercolor[fg]{normal text}\normalsize\itshape
	}{\end{frame}}

%%% A itemframe defines a list of items
\newenvironment{itemframe}[1]
{
	\begin{frame}
		

		\usebeamerfont{title}\usebeamercolor[fg]{box title}\Large #1 \\
		\vskip0.7cm
		\usebeamercolor[fg]{normal text}\normalsize\itshape
		
	}{\end{frame}}

\newcommand{\emptyslide}{\begin{frame}[plain]\end{frame}}

% -----------------------------------

\subject{Première Guerre Mondiale}
\title{Techniques et progrès de la médecine de guerre en 14-18}
%\subtitle{Rapport d'exposé}
\date{}
\author{\small Morine PINOT \& Christopher JACQUIOT}
\institute{Université de Caen}

\begin{document}
	\beamertemplatenavigationsymbolsempty
	%1ere de couverture
	\titleframe
	
	% introduction
	
	\section{Introduction}
	
		\begin{frame}{Introduction}
			\begin{itemize}
				\item Première guerre de cette envergure et depuis 40 ans en Europe
				\item Technologies nouvelles et non testées sur champ de bataille
				\item Médecine encore relativement peu développée en général
				\item Beaucoup de jeunes dont des médecins sont envoyés au front
				\item Beaucoup d'entre eux reviennent gravement blessés
			\end{itemize}
			
		\end{frame}
	
	% ^robematique
	
	\section{Problématique}
	
		\begin{wordframe}{De quelles façons était organisé le traitement des blessés durant la Première Guerre Mondiale?}
			
		\end{wordframe}
	
	% Sommaire
	
	\section*{Sommaire}
		
		\begin{frame}{Sommaire}
			\tableofcontents
		\end{frame}
	
	% plan detaillé
	\section{Plan détaillé}
		
		% axe 1
		
		\subsection{Les différents types de blessés}
			\begin{wordframe} {Les différents types de blessés}
			\end{wordframe}
			

		
	
			\subsubsection{Les nouvelles armes}
				\begin{itemframe}{De nouvelles armes A IMAGISER}
					\begin{itemize}
						\item Barbelés
						\item Mitraillettes lourdes et Lance-flamme
						\item Armes chimiques
						\item Artilleries
					\end{itemize}
				\end{itemframe}
			
			\subsubsection{Les nouvelles protections}
				\begin{itemframe}{De nouvelles protections A IMAGISER}
				
				\begin{itemize}
					\item Masque à Gaz
					\item Casques en métal
					\item Boucliers et blindages\footnote{Dont la "pelle-bouclier" MacAdam des canadiens}
				\end{itemize}
			 	\end{itemframe}
		
		
			\subsubsection{Les nouvelles blessures et maladies}
				\begin{itemframe}{De graves blessures}
					\begin{itemize}
						\item Brûlures graves
						\item Défigurations importantes 
						\item Maladies nerveuses \& mentales (ex Obusite)
						\item Blessures chimiques
						\item Maladies due au manque d'hygiène
					\end{itemize}
				\end{itemframe}
		
		% axe 2
		
		
		\subsection{La gestion des blessés sur le front}
			\begin{wordframe} {La gestion des blessés sur le front}
			\end{wordframe}
		
			\subsubsection{Le chemin d'un blessé}
				\begin{itemframe}{Le chemin d'un blessé A IMAGISER}
					\begin{itemize}
					\item Récupération par des brancardiers ou des camarades
					\item Redirection ou soins selon gravité des blessures au postes divisionnaires
					\item Transport par camion-ambulance sous anesthésiant vers l'HOC à 15-20km du front
					\end{itemize}
				\end{itemframe}
			
			\subsubsection{Récupération des blessés au front}
				\begin{itemframe}{Les soins près des tranchées A IMAGISER}
					\begin{itemize}
						\item Postes de secours divisionnaires dans les tranchées : tri des blessés et premier soins basiques
						\item Ambulances chirurgicales près du front : premiers soins, radiologies et chirurgies précaires
					\end{itemize}
				\end{itemframe}
			
			
			\subsubsection{Triage des blessés}
				\begin{itemframe}{Le triage des blessés A IMAGISER}
					\begin{itemize}
					\item Légèrement blessé: Pas besoin de traitements intensifs, retournent au combat aussitôt traités.
					\item Nécessite hospitalisation: Envoyé à l'arrière par ambulance à l'hôpital le plus proche.
					% La fin des haricots, aka :
					% Les carottes sont cuites, aka:
					\item Trop gravement blessé: Peu de chances de survie, mis en confort mais peu de soins prodigués, priorité aux moins blessés par manque de moyens. %\cite{triage}
				\end{itemize}
				\end{itemframe}
			
		% axe 3
		
		
		\subsection{La gestion des blessés à l'arrière}
			
			\begin{wordframe} {La gestion des blessés à l'arrière}
			\end{wordframe}
		
			\subsubsection{Logistique}
				\begin{imageframe}{pics/image-vide.jpg}{Le trajet d'un blessé selon son état}%trajet_arriere.png}
					%inserer image de:
					%(
					%T - blessés a pied
					%P - Pas mort cheval ou ambulance
					%HOC - déplaçables train
					%HB - traitement + avancé train
					%HL - traités train
					%)
					
				\end{imageframe}
			
			\subsubsection{\'Evolution de l'organisation des hôpitaux}
				\begin{itemframe}{Les pratiques évoluent}
					\begin{itemize}			
						\item Apparition de nouvelles spécialisations, domaines
						\item Optimisation de l'organisation, plus de chirurgies, moins de salle, besoin de coordination
					\end{itemize}
				\end{itemframe}
				
			
			\subsubsection{Nouvelles techniques médicales}
				\begin{itemframe}{Des avancées médicales A IMAGISER}	
					\begin{itemize}			
						\item Outils de guérison
						\item Chirurgie (Greffes, extractions sensibles, prothèses)
						\item Transfusion sanguines
					\end{itemize}
				\end{itemframe}
				
	% conclusion
	\section{Conclusion}
		\begin{itemframe}{Conclusion}
			De quelles façons était organisé le traitement des blessés durant la Première Guerre Mondiale?
			\begin{itemize}
				\item Gestion pragmatique des blessés dans les tranchées 
				\item Des progrès médicaux et technologiques forcés par la forte demande de soins
				\item Nouvelles spécialisations médicales dédiées à de nouveaux types de blessures
			\end{itemize}
		\end{itemframe}

\end{document}
