\documentclass[a4paper, BCOR=0mm, fontsize=12pt, titlepage=firstiscover]{scrreprt}

\usepackage[utf8]{inputenc}
\usepackage{hyperref}
\usepackage{graphicx}
\usepackage[nomain, acronym, xindy, toc]{glossaries}
\usepackage[xindy]{imakeidx}
\usepackage[nottoc, notlof, notlot]{tocbibind}
\usepackage{tocloft}
\usepackage[margin=1in]{geometry}
\usepackage[T1]{fontenc}
\usepackage{mathptmx}

%--------------------------------------
% Creation des entrées du glossaire
% acronymes
\newacronym{hopital}{HOC}{Hôpital d'Orientation des \'Etapes}
% mots importants
\newglossaryentry{secoursdiv}{name={poste de secours divisionnaire},description={Poste de secours avancé situé entre la première ligne du front appelée ligne de tir et la seconde appelée ligne de soutien.\cite{trenches} Il était principalement destiné à catégoriser les soins nécessaires pour rediriger les blessés vers les centres médicaux les plus adaptés ainsi qu'à promulguer les premiers soins les plus basiques.}}
	
\newglossaryentry{brancardier}{name={brancardier},description={à faire}}
%--------------------------------------
%\renewcommand{\thechapter}{}
%\renewcommand{\cftchapaftersnum}{}
\renewcommand{\thesection}{\Roman{section}}
%\renewcommand{\thesubsection}{\alph{subsection}}
%--------------------------------------
\fontfamily{Times}\fontseries{m}\selectfont
\renewcommand{\familydefault}{Times}
%--------------------------------------
% Mise des espacements des lignes à 1 et demi
% 1.6 serait du double lignes
\linespread{1.3}
%--------------------------------------
\makeglossaries 
\makeindex
%--------------------------------------
% Formatting de la page de garde
\titlehead{
	\begin{minipage}[t]{10cm}
		\flushleft
		\includegraphics[width=0.3\linewidth]{pics/logo.png}
	\end{minipage}
	\hfill
	\begin{minipage}[t]{7cm}
		\flushright
		Université de Caen, Normandie
	\end{minipage}
}

\subject{Première Guerre Mondiale}
\title{Techniques et progrès de la médecine de guerre en 14-18}
%\subtitle{Rapport d'exposé}
\date{}
%\author{Morine PINOT \& Christopher JACQUIOT}
\publishers{
	\begin{minipage}[t]{7cm}
		\flushleft
		Licence informatique 
	\end{minipage}
	\hfill
	\begin{minipage}[t]{7cm}
		\flushright
		M. PINOT \& C.JACQUIOT
	\end{minipage}
}

%---------------------------------

\begin{document}
	
	% Compilation de la premiere de couverture 
	
	\maketitle
	
	% Compilation de la table des matières 
	\tableofcontents
	
	
	
	\chapter{Introduction}
	% 10-12 lignes
	\paragraph{Contexte, situation du sujet}
	\paragraph{Contexte, expose les enjeux} test\\\\
	
	%2-3 lignes
	
	{\centering \large \textbf{Problématique}}
	\\
	%5-6 lignes
	\paragraph{Annonce du plan, 1-2 phrase pour annoncer chaque partie}
	
	
	\chapter{Plan détaillé}
	\section{Les différents types de blessés}
	\paragraph{intro 2-3 lignes pour présenter l'idée directrice}
		TODO a deux
		\subsection{Par balle}
		\subsection{Maladies et infections}
		\subsection{Par artillerie}
		\subsection{Escouades et stratégies militaires}
	
	\newpage
	\section{La gestion des blessés sur le front}
		\paragraph{Des blessés sévères et en surnombre}
		Lors de cette guerre, de nouvelles armes et de nouveaux genres de blessures jamais vues auparavant ont été découvertes.
		Jamais autant de blessures aussi sévères n'avaient été subies en aussi grand nombre.
		Comment les médecins de guerres se sont ils adaptés à la gestions des blessés venant du front?
		
		\subsection{Difficulté d'avoir des secours adéquats}
			\begin{itemize}
				\item Des médecins non préparés à ces nouvelles blessures de guerre: stages de formation nécessaires\cite{manquedeffectif}
				\item Manque d'effectifs parmi les médecins de guerre: certains ont aussi été mobilisés\cite{manquedeffectif}
			\end{itemize}
		\subsection{Le chemin d'un blessé}
			% begin - end = un environnement particulier pour un style d'affichage particulier
			% ici itemize = liste a puce
			% chaque \item represente le debut d'une nouvelle puce
			\begin{itemize}
				\item Récupération par des \glspl{brancardier} ou des camarades\cite{medicineintrenches}
				\item Redirection ou soins selon gravité des blessures au \gls{secoursdiv}\cite{medicineintrenches}
				\item Transport par camion-ambulance sous anesthésiant vers l'\Gls{hopital}\footnote{Appelé Casualty Clearing Station (CCS) ou "station de gestion des victimes" pour l'armée britannique}\cite{brancardiers}\cite{medicineintrenches} à 15-20km du front pour y être soigné/opéré

			\end{itemize}
		\subsection{Sur place}
			\begin{itemize}
				\item Postes de secours divisionnaires dans les tranchées : tri des blessés et premier soins basiques
				\item Ambulances chirurgicales près du front : premiers soins, radiologies et chirurgies précaires
			\end{itemize}
		\subsection{Triage}
			\begin{itemize}
				\item Légèrement blessé: Pas besoin de traitements intensifs, retournent au combat aussitôt traités.
				\item Nécessite hospitalisation: Envoyé à l'arrière par ambulance à l'hôpital le plus proche.
				\item Sans espoir: Peu de chances de survie, mis en confort mais sans peu de soins donnés, priorité aux moins blessés. \cite{triage}
			\end{itemize}
	\newpage
	\section{La gestion des blessés à l'arrière}
		\paragraph{intro 2-3 lignes pour présenter l'idée directrice}
		\subsection{Logistique}
		\subsection{etc}
		TODO: momo
	\chapter{Conclusion}
		\paragraph{Réponse synthétique, 1 para de 10-12 lignes}

	\printindex
	\glossarystyle{altlist}
	\printglossary[title=Glossaire et abbréviations,toctitle=Glossaire et abbréviations]
	
	\bibliographystyle{plain}
	\bibliography{expose}
	
	\chapter{Résumé}
		\paragraph{Résumé de 160 mots}
		
		\footnotetext{
			4-5 Mot-clés:
			 \textbf{mot 1},
			 \textbf{mot 2},
			 \textbf{mot 3}, 
			 \textbf{mot 4}, 
			 \textbf{mot 5}}
		
	
	
\end{document}